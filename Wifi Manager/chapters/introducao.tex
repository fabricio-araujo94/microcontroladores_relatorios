\section{Introdução}

A décima quarta atividade da disciplina de Microcontroladores consiste em criar um gerenciador de Wi-Fi para o ESP32 utilizando a biblioteca AsyncWebServer.

Um gerenciador de Wi-Fi é muito interessante para o ESP32 para que seja possível conectar o microcontrolador a alguma rede sem ser necessária informar as credenciais dentro do próprio código. Além de que uma rede pode se tornar indisponível em algum momento e será necessário conectar o ESP32 a alguma outra rede.

As credenciais são guardadas em arquivos no próprio ESP32. Ou seja, foi necessário a utilização do gerenciador de arquivos Little FS.

O projeto é bastante simples, pois para comprovar o resultado, não foi necessária utilizar nenhum outro componente. Ou seja, não foi sequer utilizada a protoboard.
