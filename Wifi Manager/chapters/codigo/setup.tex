\subsection{Função setup}

Inicializamos o \textit{Serial} com a taxa de 9600.

\begin{lstlisting}
void setup() {
    Serial.begin(9600);
\end{lstlisting}

Inicializamos o Little FS, lemos os arquivos que estão salvos no ESP32 e salvamos nas variáveis.

\begin{lstlisting}
initLittleFS();

ssid = readFile(LittleFS, ssidPath);
pass = readFile(LittleFS, passPath);
ip = readFile(LittleFS, ipPath);
gateway = readFile(LittleFS, gatewayPath);

Serial.println(ssid);
Serial.println(pass);
Serial.println(ip);
Serial.println(gateway);
\end{lstlisting}

A exibição das páginas funcionam dessa maneira: se o ESP32 estiver conectado a um rede, é exibido a página \textit{index}; e se não estiver, é exibido a página do Wi-Fi Manager.

Para a exibição de cada página, precisamos utilizar o método \textit{serverStatic} para dizer que os arquivos de um certo diretório são arquivos estáticos. Salvamos os arquivos HTML e CSS na raiz do gerenciador de arquivos, e a pasta raiz se chama "LittleFS".

\begin{lstlisting}
server.serveStatic("/", LittleFS, "/");
\end{lstlisting}

Essa é a implementação para quando a conexão estiver feita.

\begin{lstlisting}
if (initWiFi()) {
    // envia a pagina index
    server.on("/", HTTP_GET, [](AsyncWebServerRequest *request) {
      request->send(LittleFS, "/index.html", "text/html");
    });
    server.serveStatic("/", LittleFS, "/");
    
    server.begin();
\end{lstlisting}

O caso da conexão não ter sido feita é mais trabalhosa, pois precisamos que o ESP32 se torne um \textit{access point} para que outro dispositivo possa se conectar a ele. Isso é possível utilizando o método \textit{SOFTAP} do módulo WiFi, onde definimos o SSID e uma senha. No nosso projeto, utilizamos NULL para que não seja necessário utilizar uma senha.

Também precisamos informar ao usuário qual o endereço IP que ele deve acessar para entrar na página do Wifi Manager. Conseguimos o IP com o meétodo \textit{softAPIP} e imprimimos no monitor.

\begin{lstlisting}
Serial.println("Setting AP (Access Point)");
WiFi.softAP("ESP-WIFI-MANAGER", NULL);

IPAddress IP = WiFi.softAPIP();
Serial.print("AP IP address: ");
Serial.println(IP);
\end{lstlisting}

Enviamos a página HTML \textit{WIFIMANAGER}.

\begin{lstlisting}
server.on("/", HTTP_GET, [](AsyncWebServerRequest *request) {
    request->send(LittleFS, "/wifimanager.html", "text/html");
});
server.serveStatic("/", LittleFS, "/");
\end{lstlisting}

No código HTML, quando o usuário apertar o botão de \textit{submit}, os dados irão para a mesma URL só que com o método POST, e é nessa parte que pegamos os dados e escrevemos em arquivos no ESP32.

Com o método \textit{params}, conseguimos a quantidade de parâmetros que foram enviados.

\begin{lstlisting}
server.on("/", HTTP_POST, [](AsyncWebServerRequest *request) {
  int params = request->params();
\end{lstlisting}

Utilizamos um laço \textit{for} para pegar todos os parâmetros e escrever em um arquivo. Conseguimos resgatar os parâmetros utilizando o método \textit{getParam} e passando a sua posição. Ele retorna um ponteiro da classe AsyncWebParameter. Utilize o método \textit{isPost} para conferir se o parâmetro é do tipo POST.

\begin{lstlisting}
for (int i = 0; i < params; i++) {
    const AsyncWebParameter *p = request->getParam(i);
    if (p->isPost()) {
\end{lstlisting}

Precisamos conferir qual é o nome do parâmetro comparando com as quatro opções de parâmetros que existem. Em seguida, copiamos seu conteúdo para sua variável correspondente e utilizamos a função \textit{writeFile} para escrever o parâmetro num arquivo. Esse processo se repete com cada parâmetro

\begin{lstlisting}
// ssid
if (p->name() == PARAM_INPUT_1) {
    ssid = p->value().c_str();
    Serial.print("SSID set to: ");
    Serial.println(ssid);
    writeFile(LittleFS, ssidPath, ssid.c_str());
}
\end{lstlisting}

Um adendo que essa variável a qual o conteúdo é copiado é a variável global que foi criada no começo do código. É necessário para a conexão Wi-Fi que será realizada.

Feito tudo isso, informamos que o processo e reiniciamos o ESP32.

\begin{lstlisting}
request->send(200, "text/plain",
                "Done. ESP will restart, connect to your router and go to "
                "IP address: " +
                ip);
delay(3000);
ESP.restart();
\end{lstlisting}

Por fim, iniciamos o server.

\begin{lstlisting}
server.begin();
\end{lstlisting}