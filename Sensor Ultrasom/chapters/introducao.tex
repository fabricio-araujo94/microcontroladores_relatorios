\section{Introdução}

A décima segunda atividade prática da disciplina de Microcontroladores consiste em calcular a distância de um certo ponto para um objeto utilizando um sensor ultrasônico, ou apenas sonar, do modelo HC-SR04.

O sensor utiliza um transmissor para emitir ondas que irão colidir com algum objeto próximo à sua frente e irão refletir de volta para o sensor que coletará essas ondas com um receptor. Assim, com um cálculo, será possível descobrir a distância.

Para melhorar o teste, improvisamos um sensor de ré de um carro utilizando um LED RGB, que emitirá uma luz verde, amarela e vermelha para o quão próximo o carro estiver de um objeto, como uma parede ou outro carro.

