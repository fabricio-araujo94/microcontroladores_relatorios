\section{Introdução}

A décima terceira atividade da disciplina de Microcontroladores visa utilizar um potenciômetro para entender o controle de tensão em um circuito pelo movimento de seu eixo.

Para ser mais exato, o controle que o potênciometro provê é sobre a resistência do circuito. Pode-se dizer que o potênciometro é um resistor variável. Quando ele estiver no seu máximo, a tensão é mínima. Se estiver no mínimo, a tensão é máxima.

Para testar esse dispositivo, iremos imprimir em um gráfico no Thinkspeak os valores alterados na medida em que movemos o eixo. Como também, iremos controlar a luminosidade de um LED com o potenciômetro. 

