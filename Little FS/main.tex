%%%%%%%%%%%%%%%%%%%%%%%%%%%%%%%%%%%%%%%%%%%%%%%%%%%%%%%%%%%%%%%%%%%%%%
% How to use writeLaTeX: 
%
% You edit the source code here on the left, and the preview on the
% right shows you the result within a few seconds.
%
% Bookmark this page and share the URL with your co-authors. They can
% edit at the same time!
%
% You can upload figures, bibliographies, custom classes and
% styles using the files menu.
%
%%%%%%%%%%%%%%%%%%%%%%%%%%%%%%%%%%%%%%%%%%%%%%%%%%%%%%%%%%%%%%%%%%%%%%

\documentclass[12pt]{article}

\usepackage{sbc-template}

\usepackage{graphicx,url}

%\usepackage[brazil]{babel}   
\usepackage[utf8]{inputenc} 

\usepackage{float}
\usepackage{hyperref}

\usepackage{listings}
\usepackage{color}

\definecolor{dkgreen}{rgb}{0,0.6,0}
\definecolor{gray}{rgb}{0.5,0.5,0.5}
\definecolor{mauve}{rgb}{0.58,0,0.82}

\lstset{frame=none,
  language=C++,
  aboveskip=3mm,
  belowskip=3mm,
  showstringspaces=false,
  columns=flexible,
  basicstyle={\small\ttfamily},
  numbers=none,
  numberstyle=\tiny\color{gray},
  keywordstyle=\color{blue},
  commentstyle=\color{dkgreen},
  stringstyle=\color{mauve},
  breaklines=true,
  breakatwhitespace=true,
  tabsize=3
}

     
\sloppy

\title{Little FS}

\author{Fabricio Araújo Dias}

\address{Instituto Federal de Educação, Ciência e Tecnologia do Ceará
  (IFCE)\\
  Avenida Vice-Presidente José Alencar, S/N -- 61.939-140 -- Maracanaú -- CE -- Brasil
  \email{fabricio.araujo61@aluno.ifce.edu.br}
}

\begin{document} 

\maketitle

\begin{abstract}
  This report shows the step-by-step process for testing the Little FS file manager for the ESP32. The test consists of writing the credentials of a Wi-Fi network and the states of an RGB LED in \textit{.txt} files. When the ESP32 starts up, it reads these files and makes the settings.
\end{abstract}
     
\begin{resumo} 
  Esse relatório mostra o passo a passo para testar o gerenciador de arquivos Little FS para o ESP32. O teste consiste em escrever as credenciais de uma rede Wi-fi e os estados de um LED RGB em arquivos \textit{.txt}. Para quando o ESP32 iniciar, ele ler esses arquivos e realizar as configurações.
\end{resumo}


\section{Introdução}

A quinta atividade prática da disciplina de Microcontroladores consiste em utilizar arquivos no ESP32 utilizando a biblioteca \textit{Little FS}. Para representar o uso, iremos gravar informações em arquivos de texto e, ao reiniciar o ESP32, ler essas informações e configurar a conexão com a rede Wi-Fi e definir os estados de um LED RGB.

\section{Configuração no ESP32}

Utilizamos um ESP32 de 30 pinos, uma protoboard de 400 pontos, um LED RGB, três resistores de 390 ohms, quatro cabos macho-fêmea e um cabo microUSB.

Os pinos de cores do LED RGB são dispostos no topo das colunas, perto da linha negativa da protoboard para que possamos conectar o cátodo do LED RGB nessa linha. O cátodo poderia ser conectado em uma coluna diferente também.

Como é uma configuração simples e a protoboard é pequena, dispomos os resistores para que fiquem na mesma coluna de cada terminal do LED RGB. Isso poderia ser feito utilizando colunas diferentes, mas desse jeito, fica mais organizado.

Por fim, conectados os pinos do ESP32 na protoboard utilizando os cabos macho-fêmea. Os pinos escolhidos para cada terminal do LED RGB foram o 21, 19 e 18 para o temrinal da cor vermelho, verde e azul, respectivamente. O cátodo do LED RGB é conectado ao GND.

\section{Desenvolvimento do Código}

Para gerenciar arquivos, é necessário implementar muitas funções. Procurando economizar tempo e espaço, vamos nos limitar a cinco funções: criar diretório, listar diretório, escrever arquivo, ler arquivo e apagar arquivo.

\subsection{Definindo Constantes e Importando Bibliotecas}

Começando pela parte simples, criamos constantes para representar os pinos do LED (que foram 21, 19 e 18) e configuramos o modo dos pinos para OUTPUT. Também utilizamos constantes para guardar o SSID e a senha da rede Wi-Fi que iremos usar.

Para utilizar o LittleFS, o nosso gerenciador de arquivos, temos que importar duas bibliotecas: FS e LittleFS.

\lstinputlisting[language=C++, firstline=3, lastline=4]{code/main.cpp} 

\subsection{Iniciando o Gerenciador}

Antes, vamos na função setup para inicializar o gerenciador de arquivos LittleFS. Utilizamos o método begin com o parâmetro \textit{true} para que o sistema seja formatado caso ele ainda não tenho sido iniciado ou tenha ocorrido algum problema. Fizemos essa inicialização dentro de um if condicional para caso tenha dado algum erro e o gerenciador não foi inicializado. Sendo assim, não há porquê do programa continuar.

\lstinputlisting[language=C++, firstline=32, lastline=37]{code/main.cpp}

\subsection{Criando Diretórios}

A primeira função que criamos é a \textit{createDir} que recebe como parâmetros a variável do gerenciador de arquivos e um caminho que representa o diretório a ser criado. O gerenciador possui um método chamado \textit{mkdir} que, passando como parâmetro o caminho, ele irá criar esse diretório. O método também retorna algum valor para representar se o processo deu certo ou não. O que é ideal para colocarmos o resultado dentro de um \textit{if} e conferirmos se deu certo.

\lstinputlisting[language=C++, firstline=78, lastline=91]{code/main.cpp}

\subsection{Listando Diretórios}

\lstinputlisting[language=C++, firstline=98, lastline=98]{code/main.cpp}

Vamos para a função \textit{listDir} que irá listar os diretórios. Ela, na verdade, só serve para chamar uma outra função chamada de \textit{listDirAux} que recebe a instância de LittleFS, o caminho e o nível até onde queremos explorar. Como é um projeto simples, deixamos o nível em 1. Usamos o método \textit{open} do LittleFS  para "abrir" o caminho como um File. Conferimos se o valor não é nulo ou se ele não é um diretório.

\lstinputlisting[language=C++, firstline=102, lastline=112]{code/main.cpp}

Vamos utilizar o método \textit{openNextFile} para vasculharmos, um por um, os arquivos daquele diretório. Dentro de um \textit{while}, conferimos se a variável File é um diretório, e se sim, chamamos a função \textit{listDirAux} recursivamente e decrementados o nível. Caso não seja, só imprimimos o nome e o tamanho do arquivo.

\lstinputlisting[language=C++, firstline=114, lastline=124]{code/main.cpp}

\subsection{Escrevendo Arquivos}

\lstinputlisting[language=C++, firstline=137, lastline=137]{code/main.cpp}

A função \textit{writeFile} recebe a variável do gerenciador de arquivos, o caminho do arquivo e a mensagem a ser impressa. O método \textit{open} será usado novamente para abrir algo, e dessa vez acompanhado da constante File\_WRITE para indicar o modo de abertura. Mesmo que o arquivo não exista, ele será criado.

Conferimos se o arquivo foi aberto antes de imprimirmos a mensagem usando o método \textit{print} do LittleFS. O método também retorna um valor para representar se a escrita foi realmente efetuada, e aproveitamos para tratar possíveis erros. Lembre-se que desse modo, ele sempre escreve o novo valor por cima. Ao final da função, fechamos o arquivo com o método \textit{close}.

\lstinputlisting[language=C++, firstline=147, lastline=155]{code/main.cpp}

\subsection{Deletando Arquivos}

\lstinputlisting[language=C++, firstline=210, lastline=210]{code/main.cpp}

Para excluir, \textit{deleteFile} recebe a variável do LittleFS e o caminho do arquivo a ser excluído. O método \textit{remove} do FS faz a exclusão do arquivo. Novamente, o método retorna um valor para representar se o arquivo foi realmente apagado, e tratamos o problema.

\lstinputlisting[language=C++, firstline=213, lastline=220]{code/main.cpp}

Se nada der errado na hora de gravar os documentos, não iremos usar essa função, mas fica registrado caso precise.

\subsection{Ler Arquivos}

\lstinputlisting[language=C++, firstline=158, lastline=158]{code/main.cpp}

Essa parte é mais trabalhosa, pois precisamos não só ler o arquivo como também guardar as informações que estão nele. A função recebe a variável do gerenciador de arquivos e o caminho do arquivo.

Primeiro, instanciamos um vetor de strings vazio chamado de \textit{parts}. Os vetores não são nativos da biblioteca padrão do C++, então você precisa importar a biblioteca \textit{vector}.

\lstinputlisting[language=C++, firstline=5, lastline=5]{code/main.cpp} \newpage

De novo, utilizamos o método \textit{open} para abrir o arquivo no caminho especificado. Checamos se realmente abrimos um arquivo ou não. Se não, retornamos o vetor vazio. Como salvamos o arquivo numa variável do tipo File, podemos usar o método \textit{readString} para ler todo o conteúdo do arquivo e salvar como uma String. Fechamos o arquivo com \textit{close}.

\lstinputlisting[language=C++, firstline=164, lastline=171]{code/main.cpp}

Chegamos à parte complicada que é dividir a string que possuímos para separar as informações que iremos utilizar. Para facilitar o trabalho, vamos definir que as informações estão separadas por vírgula e sem espaço. Então, implementados uma função extra para fazer essa repartição chamada splitString, que recebe a String e o delimitador.

\lstinputlisting[language=C++, firstline=182, lastline=182]{code/main.cpp}

Em splitString, instanciamos um novo vetor de Strings vazio chamado \textit{words}. Também criamos duas variáveis do tipo inteiro chamadas de start (que é inicializada em 0) e end. 

\lstinputlisting[language=C++, firstline=184, lastline=187]{code/main.cpp}

Fizemos uso de um while. A condição dele é que \textit{end} seja diferente de $-1$. A cada iteração, \textit{end} recebe o index do delimitador na string dado um ponto inicial \textit{start}. Descobrindo isso com o método \textit{indexOf} da classe String. Se ele não encontrar o delimitador, ele retorna $-1$. 

Dentro do While, iremos fazer o uso de outro método da classe String chamada \textit{substring} que retorna uma string dado um ponto inicial e um ponto final. Salvamos essa substring e adicionamos ao vetor \textit{words} utilizando o método \textit{push\_back} da classe \textit{vector}. Atualizamos o valor de \textit{start} somando \textit{end} mais o tamanho do delimitador (em casos que ele tenha mais de um caractere).

\lstinputlisting[language=C++, firstline=189, lastline=194]{code/main.cpp} 

Quando ele sair desse \textit{while}, ainda faltará uma palavra a ser adicionada, então só utilizamos novamente o método \textit{substring} para recolher essa última palavra e salvá-la no vetor \textit{words}. Pode acontecer do delimitador ser o último caractere da string, por isso confira se essa última substring tenha um tamanho maior do que 0.

\lstinputlisting[language=C++, firstline=196, lastline=200]{code/main.cpp}

Por fim, retorne o vetor para readFile, e de readFile, retorne esse vetor também para que as informações contídas nele sejam usadas em outro lugar.

\subsection{Retornando ao Setup}

Primeiro, criamos dois diretórios utilizando a função \textit{createDir}: \textit{wifi} e \textit{rgb\_state}. Os títulos são autoexplicativos.

\lstinputlisting[language=C++, firstline=38, lastline=39]{code/main.cpp}

Em seguida, criamos strings que irão guardar as informações para enviarmos ao ESP32. A string \textit{credentials} guarda o SSID e a senha da rede a qual iremos conectar. A string \textit{states} guarda os estados dos pinos do LED RGB. Vamos adotar uma ordem específica e clara para usarmos as informações depois. A string \textit{credentials} começa com o SSID e depois vem a senha. A string \textit{states} começa com o estado do LED vermelho, depois vem o verde, e por fim o azul. As informações são separadas por vírgulas e não devem possuir espaço.

\lstinputlisting[language=C++, firstline=41, lastline=42]{code/main.cpp}

Com o conteúdo guardado, vamos gravar num arquivo de texto no ESP32 utilizando a função writeFile. Passamos o endereço absoluto, especificando o diretório e o nome do arquivo com a extensão .\textit{txt}. Não foi dito, mas a nossas funções normalmente recebem parâmetros \textit{const char *} para textos, mas existe um método da clásse String chamado \textit{c\_str} que retorna o conteúdo da string em \textit{const char *}.

\lstinputlisting[language=C++, firstline=44, lastline=45]{code/main.cpp}

Nesse momento, fazemos \textit{upload} desse código para o ESP32 utilizando o comando $Alt + Ctrl + U$. Agora que temos as informações gravadas no ESP32, apagamos as diretivas que escrevemos no \textit{setup} até agora. Deixando apenas a configuração dos modos dos pinos e a inicialização do LittleFS. \newpage

Vamos nos preocupar em recuperar as informações. Vamos utilizar a função readFile para ler os arquivos que criamos passando o caminho deles. A função retorna um vetor de strings, então criamos um vetor chamado \textit{wifi\_creds} e outro chamado \textit{rgbs}.

\lstinputlisting[language=C++, firstline=47, lastline=48]{code/main.cpp}

Para conectar a internet, primeiro precisamos importar a biblioteca WiFi. 

\lstinputlisting[language=C++, firstline=4, lastline=4]{code/main.cpp}

E configuramos o módulo Wifi para funcione em modo estação. Para que possamos nos conectar a um \textit{access point}. Em seguida, utilizamos o método \textit{begin} para nos conectar à internet passando o SSID e a senha da rede. Para acessar o valor de uma posição do vetor, utilizamos o método \textit{at} e o index. Lembrando que, segundo nossa ordem, a primeira posição é o SSID e a segunda é a senha. O método \textit{begin} só aceita \textit{const char *}, então faremos novamente o uso do método \textit{c\_str}.

\lstinputlisting[language=C++, firstline=50, lastline=51]{code/main.cpp}

Criamos um laço \textit{while} para que confira se a conexão foi realizada com sucesso. Para que o \textit{while} não seja infinito, usamos uma variável wifiRetries que incrementa a cada iteração e irá encerrar o \textit{while} quando chegar em 30.

\lstinputlisting[language=C++, firstline=53, lastline=59]{code/main.cpp}

E por último, vamos alterar o estado dos pinos do LED RGB. Lembrando da ordem das informações em que primeiro vem o estado do pino vermelho, depois do verde, e por último, o azul. Normalmente, utilizamos a função \textit{digitalWrite}. No primeiro parâmetro, dizemos qual é o pino, e no segundo parâmetro, o seu estado. No pino vermelho, acessamos a primeira string do vetor utilizando o método \textit{at} e comparamos se a string é igual à \textit{"HIGH"}. Utilizamos um operador ternário para caso for, ele retorna \textit{HIGH}, e se não for, retorne \textit{LOW}.

\lstinputlisting[language=C++, firstline=69, lastline=71]{code/main.cpp}

Acontece que \textit{HIGH} e \textit{LOW} são constantes de inteiros e não strings. \textit{HIGH} é $1$ e \textit{LOW} é $0$. Poderiamos salvar números no arquivo, e no \textit{digitalWrite}, fazemos um \textit{casting} neles, mas dessa fica mais claro na hora de escrever e ler.

É possível conferir o código implementado nesse \href{https://github.com/fabricio-araujo94/microcontroladores/tree/main/little_fs}{repositório no GitHub}.

\section{Considerações Finais}

E então, conectamos o ESP32 ao computador utilizando um cabo microUSB e enviamos o código para o microcontrolador utilizando novamento o comando $Alt + Ctrl + U$.

Assim, ao iniciar o ESP32, ele lerá os arquivos e irá configurar automaticamente o Wi-fi e os estados dos pinos do LED RGB mesmo que não passamos explicitamente as informações no código. Por certas razões, não foi possível gravar e nem tirar fotos da prática funcionando.


\end{document}