%%%%%%%%%%%%%%%%%%%%%%%%%%%%%%%%%%%%%%%%%%%%%%%%%%%%%%%%%%%%%%%%%%%%%%
% How to use writeLaTeX: 
%
% You edit the source code here on the left, and the preview on the
% right shows you the result within a few seconds.
%
% Bookmark this page and share the URL with your co-authors. They can
% edit at the same time!
%
% You can upload figures, bibliographies, custom classes and
% styles using the files menu.
%
%%%%%%%%%%%%%%%%%%%%%%%%%%%%%%%%%%%%%%%%%%%%%%%%%%%%%%%%%%%%%%%%%%%%%%

\documentclass[12pt]{article}

\usepackage{sbc-template}

\usepackage{graphicx,url}

%\usepackage[brazil]{babel}   
\usepackage[utf8]{inputenc} 

\usepackage{float}
\usepackage{hyperref}

\usepackage{listings}
\usepackage{color}

\definecolor{dkgreen}{rgb}{0,0.6,0}
\definecolor{gray}{rgb}{0.5,0.5,0.5}
\definecolor{mauve}{rgb}{0.58,0,0.82}

\lstset{frame=none,
  language=C++,
  aboveskip=3mm,
  belowskip=3mm,
  showstringspaces=false,
  columns=flexible,
  basicstyle={\small\ttfamily},
  numbers=none,
  numberstyle=\tiny\color{gray},
  keywordstyle=\color{blue},
  commentstyle=\color{dkgreen},
  stringstyle=\color{mauve},
  breaklines=true,
  breakatwhitespace=true,
  tabsize=3
}

     
\sloppy

\title{Acendendo diferentes cores LED RGB (PWM)}

\author{Fabricio Araújo Dias}

\address{Instituto Federal de Educação, Ciência e Tecnologia do Ceará
  (IFCE)\\
  Avenida Vice-Presidente José Alencar, S/N -- 61.939-140 -- Maracanaú -- CE -- Brasil
  \email{fabricio.araujo61@aluno.ifce.edu.br}
}

\begin{document} 

\maketitle

\begin{abstract}
    This report describes the execution of the eleventh practical activity of the Microcontrollers course, which makes use of the Pulse Width Modulation technique to display different colors on an RGB LED.
\end{abstract}
     
\begin{resumo}
    Esse relatório descreve a execução da décima primeira atividade prática da disciplina de Microcontroladores que faz uso da técnica de Modulação por Largura de Pulso para mostrar cores diversas num LED RGB.
\end{resumo}

\section{Introdução}

A décima primeira atividade prática da disciplina de Microcontroladores consiste em testar a Modulação por Largura de Pulso (PWM). A técnica nada mais é do que uma forma de controlar a potência e a tensão.

Estabelecendo um período, podemos controlar até que momento se permite a passagem de potência, para depois cessar. Dessa forma, se a potência é aplicada em 50\% do período, então só é aplicada 50\% da potência.

Para realizar um teste, usamos a técnica de PWM para definir cores diferentes em um LED RGB, como as cores amarelo, roxo e laranja. Ou seja, aplicamos diferentes potências em cada pino do LED RGB para criar essas cores.

\section{Configuração no ESP32}

Não foi necessário a utilização de nenhum componente além do ESP32. Com exceção, é claro, de algum aparelho eletrônico que consegue se conectar a qualquer \textit{access point}.

\section{Desenvolvimento do código}
Para desenvolver o código, utilizamos o Visual Studio Code (VSCode) junto com a extensão do Platform IO. A extensão nos dá acesso à página inicial do PlatformIO. Criamos um projeto chamado "pwm\_", definimos que a placa é Espressif ESP32 Dev Module e que o framework a ser utilizado é o Arduino.

Com o projeto criado, acessamos o arquivo platformio.ini e definimos que a taxa do monitor é de 9600. Para então, irmos ao arquivo main.cpp e iniciarmos o Serial com a taxa que escolhemos na função \textit{setup}.

\begin{lstlisting}
// platformio.ini
monitor_speed = 9600

// main.cpp
void setup() {
    Serial.begin(9600);
}

\end{lstlisting}

No início do código, definimos constantes para que guardem o valor dos pinos escolhidos.

\begin{lstlisting}
#define RED 18
#define GREEN 19
#define BLUE 21
\end{lstlisting}

Em \textit{setup}, configuramos o modo dos pinos para que sejam de saída.

\begin{lstlisting}
pinMode(RED, OUTPUT);
pinMode(GREEN, OUTPUT);
pinMode(BLUE, OUTPUT);
\end{lstlisting}

Para definir a intensidade do brilho de cada LED, usamos a função \textit{analogWrite} que recebe como parâmetros o número do pino e o valor do \textit{duty cycle}, que vai de 0 até 255. O \textit{duty cycle} é a largura do pulso da onda. Para cada terminal do LED RGB, iremos usar essa função.

\begin{lstlisting}
void analogWrite(uint8_t pin, int value);
\end{lstlisting}

Escolhemos as cores amarelo, roxo e laranja para testar essa funcionalidade. Para saber os valores, visitamos o site \href{https://htmlcolorcodes.com/}{htmlcolorcodes.com} que fornece um \textit{color picker} e mostra como essa cor pode ser definida em diferentes formados, como o RGB.

\begin{figure}[H]
    \centering
    \includegraphics[width=0.5\linewidth]{img/htmlcolors.jpg}
    \caption{Site HTML Color Codes}
    \label{fig:html-color-codes-site}
\end{figure}

Recolhendo as informações e aplicando nas funções, o código fica dessa forma.

\begin{lstlisting}
void loop() {
    // yellow
    analogWrite(RED, 255);
    analogWrite(GREEN, 255);
    analogWrite(BLUE, 0);
    
    // purple
    analogWrite(RED, 158);
    analogWrite(GREEN, 10);
    analogWrite(BLUE, 149);
    
    // orange
    analogWrite(RED, 251);
    analogWrite(GREEN, 64);
    analogWrite(BLUE, 3);
}
\end{lstlisting}

Adicionamos um \textit{delay} de três segundos após cada configuração para que seja possível visualizar cada uma das cores.

\begin{lstlisting}
delay(3000);
\end{lstlisting}


\section{Considerações Finais}

Enviamos o código para o ESP32 utilizando o comando \textit{Alt + Ctrl + U}. Em certos casos, é preciso apertar o botão de BOOT para enviar. Feito o envio, as cores devem aparecer no LED RGB.

\begin{figure}
    \centering
    \includegraphics[width=0.5\linewidth]{img/final_result.jpg}
    \caption{Resultado final.}
    \label{fig:final-result}
\end{figure}

Para conferir o código completo, acesse esse \href{https://github.com/fabricio-araujo94/microcontroladores/tree/main/pwm_}{link} para o repositório no GitHub. Para ver o resultado da prática em execução, assista esse \href{https://youtu.be/cyvFmMO40j8}{vídeo} no YouTube.

\end{document}