\section{Introdução}

A décima primeira atividade prática da disciplina de Microcontroladores consiste em testar a Modulação por Largura de Pulso (PWM). A técnica nada mais é do que uma forma de controlar a potência e a tensão.

Estabelecendo um período, podemos controlar até que momento se permite a passagem de potência, para depois cessar. Dessa forma, se a potência é aplicada em 50\% do período, então só é aplicada 50\% da potência.

Para realizar um teste, usamos a técnica de PWM para definir cores diferentes em um LED RGB, como as cores amarelo, roxo e laranja. Ou seja, aplicamos diferentes potências em cada pino do LED RGB para criar essas cores.